\addsec*{Abstract}
Simulations engineering, known for its complex and multidisciplinary nature, requires accurate and efficient methods for configuration management. It is also sometimes difficult for novice simulation engineers to carry out the configuration tasks of new simulations. The idea therefore arose of using the data from previous simulations to set up an assistance platform, which could make the task of systems engineers easier. The proposed platform aims to identify previous simulations akin to the one under consideration and provide plausible parameter values for the new simulation. Leveraging semantic technologies, grounded in formal ontologies and knowledge representation, emerges as a promising approach to address the intricacies associated with managing simulation data. This thesis conducts a feasibility study on implementing semantic technologies to improve simulation configuration processes. It starts by establishing a theoretical foundation on semantic technology's principles and its relevance in systems engineering, with a focus on enhancing accuracy, traceability, and adaptability. The outcomes of this study are anticipated to offer valuable insights into the viability of employing semantic technologies to enhance simulation configuration processes in systems engineering. Furthermore, this research establishes a foundational framework for future exploration and development in the realm of semantic technology, with a particular emphasis on advancing the configurability and efficiency of simulations within complex systems.\\

\textbf{KEYWORDS}: Semantic Web, Ontologies, Case-Based Reasoning, Similarity, Simulation, Processes, Expert-System.

