\section{Evaluation\label{sec:evaluation}}
In this chapter, we present a comprehensive evaluation of the solution developed to support configuration processes. The evaluation aims to determine the effectiveness, accuracy, ease of use and overall performance of the platform, based on feedback from potential users.\\


\subsection{Methodology}
We used a multi-faceted approach to evaluate the platform, combining quantitative and qualitative methods. The main source of data was a Google form survey distributed to a variety of potential users ranging from novices to automotive industry professionals performing simulation configuration tasks. The survey comprises twenty-one questions covering various aspects of platform functionality, user experience and adoption potential.\\


% Is the result good? How efficient?
\subsection{Participant Demographics}
The survey gathered responses from a diverse group of automotive industry participants, including engineers, researchers and managers. Respondents demonstrated a balanced spread of experience, with 35\% from 0 to 2 years, 30\% from 3 to 5 years, 20\% from 6 to 10 years and 15\% with more than 10 years' experience. This diversity is crucial to understanding how easy it is to use the platform at different levels of expertise.\\


\subsection{Platform Usage and Frequency}
The results showed that a large proportion of participants (78\%) had previous experience of simulation configuration platforms, and that 65\% frequently performed simulation configuration tasks. This created a user base with a reasonable level of expertise in the field.\\


\subsection{User Interface and Experience}
Participants rated the platform's user interface with an average score of 4.2 out of 5, indicating a generally positive perception. In addition, 82\% of users found navigation on the platform intuitive, confirming its clarity and ease of use.\\


\subsection{Effectiveness and Accuracy}
The platform's effectiveness in identifying similar simulations based on key attributes received an average score of 4.4 out of 5. In addition, 88\% of respondents said that the platform's suggestions for parameter values were relevant to their simulation configuration needs. This suggests a promising capability for knowledge graph-based identification.\\
While the majority of users (88\%) felt that the suggested parameter values were relevant, 12\% expressed concerns about their accuracy. Further study of specific cases of inaccuracy will be essential to refine the system.\\


\subsection{Speed and Efficiency}
The platform proved its efficiency, achieving an average score of 4.3 out of 5 for the speed with which similar simulations were identified. A notable 90\% of participants experienced no delays or performance issues during their interaction with the platform. However, 10\% of respondents reported occasional delays, requiring further investigation of potential performance bottlenecks.\\
Users generally agreed (85\%) that the platform efficiently completed tasks related to simulation setup. Any reported inefficiencies will be investigated to improve the overall workflow.\\


\subsection{Usability}
Ease of use was well received, with an average score of 4.5 out of 5. This indicates that the platform's features and functionality were clear and easy to understand for the majority of users.\\
Open-ended responses provided valuable insights into user expectations and specific areas for improvement, such as enhanced visualization tools and additional customization options.\\


\subsection{Suggestions for Improvement}
Participants provided valuable suggestions for improvement, particularly with regard to visualization functions and the incorporation of additional domain-specific parameters. These ideas will guide future developments to meet user needs and preferences.\\


\subsection{Compatibility}
An encouraging 85\% of respondents confirmed that the platform would integrate seamlessly with their existing simulation tools or workflows. The remaining 15\% reported minor compatibility issues, which will be addressed in future updates.\\


\subsection{Likelihood to Adopt}
The likelihood of users adopting the platform for their simulation configuration needs was favorable, with an average score of 4.2 out of 5. Positive feedback emphasized the platform's potential to streamline configuration processes, as well as its accuracy and efficiency, as key factors influencing their decision.\\


The results of the evaluation confirm the success of the solution. The positive feedback and constructive suggestions received from users will form the basis for future iterations of the platform, which will focus on meeting specific user needs and further enhancing its capabilities. The platform shows promising potential for practical adoption in real-life scenarios in the field of automotive simulation.


